\documentclass{ctexart}

\usepackage{graphicx}
\usepackage{amsmath}

\title{作业一:带皮亚诺余项的泰勒定理问题的叙述与证明}


\author{邵盛栋 \\ 信息与计算科学 3200103951}

\begin{document}

\maketitle


这是一个来自数学分析领域的问题,如果函数$ f $在点$ x_{0} $可导,则有
\[f(x)=f(x_{0})+f'(x_{0})(x-x_{0})+o(x-x_{0})\]
即在点$ x_{0} $附近,用一次多项式$ f(x_{0})+f'(x_{0})(x-x_{0}) $逼近函数$ f(x) $时,其误差为$ (x-x_{0}) $的高阶无穷小量.然而在很多场合,取一次多项式逼近是不够的,往往需要用二次或高于二次的多项式去逼近,并要求误差为$ o((x-x_{0})^{n}) $,其中$ n $为多项式的次数.为此,我们考察任一$ n $次多项式
\begin{equation}
	p_{n}(x)=a_{0}+a_{1}(x-x_{0})+a_{2}(x-x_{0})^{2}+\cdots+a_{n}(x-x_{0})^{n}
	\label{eq:1}
\end{equation}
逐次求它在点$ x_{0} $的各阶导数,得到
\[p_{n}(x_{0})=a_{0},p_{n}'(x_{0})=a_{1},p_{n}''(x_{0})=2!a_{2},\cdots,p_{n}^{(n)}(x_{0})=n!a_{n}\]
即
\[a_{0}=p_{n}(x_{0}),a_{1}=\dfrac{p_{n}'(x_{0})}{1!},a_{2}=\dfrac{p_{n}''(x_{0})}{2!},\cdots,a_{n}=\dfrac{p_{n}^{(n)}(x_{0})}{n!}\]
由此可见,多项式$ p_{n}(x) $的各项系数由其在点$ x_{0} $的各阶导数值所唯一确定.

对于一般函数$ f $,设它在点$ x_{0} $存在直到$ n $阶的导数.由这些导数构造一个$ n $次多项式
\begin{equation}
	\begin{split}
		T_{n}(x)=&f(x_{0})+\dfrac{f'(x_{0})}{1!}(x-x_{0})+\dfrac{f''(x_{0})}{2!}(x-x_{0})^{2}+\cdots+\\
		&\dfrac{f^{(n)}(x_{0})}{n!}(x-x_{0})^{n}
		\label{eq:2}
	\end{split}
\end{equation}
称为函数$ f $在点$ x_{0} $的\textbf{泰勒多项式},$ T_{n}(x) $的各项系数$ \dfrac{f^{(k)}(x_{0})}{k!}(k=1,2,\cdots,n) $称为\textbf{泰勒系数}.由上面对多项式系数的讨论,易知$ f(x) $与其泰勒多项式$ T_{n}(x) $在点$ x_{0} $有相同的函数值和相同的直至$ n $阶导数值,即
\begin{equation}
	f^{(k)}(x_{0})=T_{n}^{(k)}(x_{0}),k=0,1,2,\cdots,n
	\label{eq:3}
\end{equation}
下面要证明$ f(x)-T_{n}(x)=o((x-x_{0})^{n}) $,即以式\ref{eq:2}所示的泰勒多项式逼近$ f(x) $时,其误差为关于$ (x-x_{0})^{n} $的高阶无穷小量.
\section{定理描述}
定理叙述如下: 若函数$ f $在点$ x_{0} $存在直至$ n $阶导数,则有$ f(x)=T_{n}(x)+o((x-x_{0})^{n}) $,即
\begin{equation}
	\begin{split}
		f(x)=&f(x_{0})+\dfrac{f'(x_{0})}{1!}(x-x_{0})+\dfrac{f''(x_{0})}{2!}(x-x_{0})^{2}+\cdots+\\
		&\dfrac{f^{(n)}(x_{0})}{n!}(x-x_{0})^{n}+o((x-x_{0})^{n})
		\label{eq:4}
	\end{split}
\end{equation}

\section{证明}
设
\[R_{n}(x)=f(x)-T_{n}(x),Q_{n}(x)=(x-x_{0})^{n}\]
现在只要证
\[\lim_{x\to x_{0}}\dfrac{R_{n}(x)}{Q_{n}(x)}=0\]

由关系式\ref{eq:3}可知,
\[R_{n}(x_{0})=R_{n}'(x_{0})=\cdots=R_{n}^{(n)}(x_{0})=0\]
并易知
\[Q_{n}(x_{0})=Q_{n}'(x_{0})=\cdots=Q_{n}^{(n-1)}(x_{0})=0,Q_{n}^{(n)}(x_{0})=n!\]
因为$ f^{(n)}(x_{0}) $存在,所以在点$ x_{0} $的某邻域$ U(x_{0}) $上$ f $存在$ n-1 $阶导函数.于是,当$ x\in U^{\circ}(x_{0}) $且$ x\to x_{0} $时,允许接连使用洛必达法则$ n-1 $次,得到
\[
\begin{split}
	\lim_{x\to x_{0}}\dfrac{R_{n}(x)}{Q_{n}(x)}&=\lim_{x\to x_{0}}\dfrac{R_{n}'(x)}{Q_{n}'(x)}=\cdots=\lim_{x\to x_{0}}\dfrac{R_{n}^{(n-1)}(x)}{Q_{n}^{n-1}(x)}\\
	&=\lim_{x\to x_{0}}\dfrac{f^{(n-1)}(x)-f^{(n-1)}(x_{0})-f^{(n)}(x_{0})(x-x_{0})}{n(n-1)\cdots2(x-x_{0})}\\
	&=\dfrac{1}{n!}\lim_{x\to x_{0}}[\dfrac{f^{(n-1)}(x)-f^{(n-1)}(x_{0})}{x-x_{0}}-f^{(n)}(x_{0})]\\
	&=0
\end{split}
\]

定理所证的式\ref{eq:4}称为函数$ f $在$ x_{0} $的\textbf{泰勒公式},$ R_{n}(x)=f(x)-T_{n}(x) $称为\textbf{泰勒公式的余项},形如$ o((x-x_{0})^{n}) $的余项称为\textbf{皮亚诺(Peano)型余项}.所以式\ref{eq:4}又称为\textbf{带有皮亚诺型余项的泰勒公式}.




\end{document}
